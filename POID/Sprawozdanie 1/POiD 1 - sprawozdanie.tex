\documentclass{classrep}
\usepackage[utf8]{inputenc}

\usepackage[pdftex]{color,graphicx}
\DeclareGraphicsExtensions{.pdf,.png,.jpg}

\usepackage{float}

\usepackage{color}

\usepackage{listings}

\usepackage{color}

\usepackage{url}

\usepackage{hyperref}

\usepackage[polish]{babel}

\usepackage{enumitem}

\usepackage{indentfirst}

\hypersetup{colorlinks=false,pdfborder={0 0 0}}


\definecolor{javared}{rgb}{0.6,0,0} % for strings
\definecolor{javagreen}{rgb}{0.25,0.5,0.35} % comments
\definecolor{javapurple}{rgb}{0.5,0,0.35} % keywords
\definecolor{javadocblue}{rgb}{0.25,0.35,0.75} % javadoc

\lstset{language=Java,
basicstyle=\ttfamily,
keywordstyle=\color{javapurple}\bfseries,
stringstyle=\color{javared},
commentstyle=\color{javagreen},
morecomment=[s][\color{javadocblue}]{/**}{*/},
numbers=left,
numberstyle=\tiny\color{black},
stepnumber=2,
numbersep=10pt,
tabsize=4,
showspaces=false,
showstringspaces=false}


\studycycle{Informatyka, studia dzienne, mgr II st.}
\coursesemester{I}

\coursename{Przetwarzanie obrazu i dźwięku}
\courseyear{2011/2012}

\courseteacher{dr inż. Arkadiusz Tomczyk}
\coursegroup{środa, 8:15}

\author{
  \studentinfo{Paweł Musiał}{178726} \and
  \studentinfo{Łukasz Michalski}{178724}
}
\title{Zadanie 1:\\  \textbf {Szkielet aplikacji do przetwarzania \\i analizy obrazów, operacje podstawowe, usuwanie szumu, modyfikacje histogramu, filtracja liniowa i nieliniowa, splot}}
\svnurl{http://serce.ics.p.lodz.pl/svn/labs/poid/at_sr0800/lmpm}

\begin{document}
\maketitle

\addtocounter{footnote}{1}

\section{Cel}

Celem zadania było stworzenie szkieletu aplikacji służącej do wczytywania, zapisywania i przede wszystkim przetwarzania obrazów, która zostanie wykorzysta zarówno w tym, jak i w następnym zadaniu. Na potrzeby pierwszego zadania w powstałym szkielecie należało:

\begin{list}{•}{}
\item zaimplementować podstawowe operacje przetwarzania obrazów: zmianę jasności i kontrastu oraz wyznaczenie negatywu obrazu;
\item zaimplementować filtr ze średnią arytmetyczną i filtr medianowy z możliwością wyboru rozmiaru maski oraz zaproponować metody obiektywnej oceny ich działania i skuteczności w usuwaniu różnego rodzaju szumu;
\item wyznaczyć i wyświetlić histogram jasności pikseli obrazu, a w przypadku
obrazów kolorowych także histogram wartości poszczególnych kanałów;
\item w oparciu o wyznaczony histogram dokonać modyfikacji obrazu w taki sposób, aby osiągnąć zadaną charakterystykę histogramu obrazu wynikowego;
\item zaimplementować możliwość filtracji liniowej opartej o splot w dwóch wersjach: na podstawie jednego z wariantów gotowych masek oraz z możliwością wyboru własnego rozmiaru maski i wartości poszczególnych jej elementów;
\item opracować i zaimplementować algorytm filtracji nieliniowej obrazu w dziedzinie czasu zgodnie z regułą zawartą w przydzielonym wariancie zadania, a także zapewnić możliwość wyboru parametrów filtru tam, gdzie jest to konieczne;
\item przemyśleć i zaimplementować sposób normalizacji wyników dla dwóch poprzednich punktów tak, aby mogły być one wyświetlone w postaci obrazu wynikowego.
\end{list}

W sprawozdaniu zamieszczono wyniki działania poszczególnych algorytmów oraz porównanie ich w różnych wariantach ustawień i~dla różnych problemów. Badania przeprowadzono zarówno na obrazach kolorowych (24-bitowych), jak i~w odcieniach szarości (8-bitowych), a~także czarno-białych (1-bitowych). 

\section{Wprowadzenie}
Aby przetworzyć obraz należy w programie przechowywać go w postaci wartości poszczególnych pikseli w obrazie. Wartości te przechowywane są na 8-bitach dla obrazu w odcieniach szarości oraz w 24-bitach (po 8 bitów na kanał) w obrazach kolorowych. Na tych wartościach wykonywane są poszczególne algorytmy (dla obrazu kolorowego na każdym kanale osobno).

\subsection{Zmiana jasności}
Zmiana jasności obrazu polega na zmniejszeniu bądź zwiększeniu składowych RGB o pewną wartość stałej b. Jeżeli jedna składowa uzyska wartość większą od dopuszczalnego zakresu to wynikiem jest wartość maksymalna z tego zakresu. Analogicznie, jeżeli składowa ma wartość mniejszą to wynikiem jest wartość minimalna. Powyższą operacją można zapisać za pomocą wzoru:
\begin{equation}
p(i)=\left\{
\begin{array}{l l l}
0 & \quad \mbox{i +b $<$ 0,} \\
i+b & \quad \mbox{0 $\leq$ i + b $\leq$  $i_{max}$,} \\
i_{max} & \quad \mbox{i + b $>$ $i_{max}$,} \\
\end{array}
\right.
\end{equation}
gdzie:
\begin{description}
\item $p(i)$ -- nowa wartość składowej RGB,
\item i -- wartość konkretnej składowej danego piksela,
\item b -- stała, o którą zmieniamy daną składową RGB,
\item $i_{max}$ -- maksymalna dopuszczalna wartość.
\end{description}

\subsection{Zmiana kontrastu}
Zmiana kontrastu polega na przekształceniu danego obrazu wykonując operację według wzoru:
\begin{equation}
\label{kontrastWzor}
p(i)=\left\{
\begin{array}{l l l}
0 & \quad \mbox{$a (i-\frac{i_{max}}{2}) + \frac{i_{max}}{2}$ $<$ 0,} \\
a (i-\frac{i_{max}}{2}) + \frac{i_{max}}{2} & \quad \mbox{0 $\leq$ $a (i-\frac{i_{max}}{2}) + \frac{i_{max}}{2}$ $\leq i_{max}$ ,} \\
i_{max} & \quad \mbox{$a (i-\frac{i_{max}}{2}) + \frac{i_{max}}{2}$ $ > i_{max}$,} \\
\end{array}
\right.
\end{equation}
gdzie:
\begin{description}
\item $p(i)$ -- nowa wartość składowej RGB,
\item i -- wartość konkretnej składowej danego piksela,
\item a -- stała, której wartość określa czy zwiększa się kontrast obrazu
\item $i_{max}$ -- maksymalna dopuszczalna wartość.
\end{description}

W przypadku jeżeli stała a przyjmuje wartości większe od 1, następuje zwiększenie kontrastu obrazu. Natomiast, jeżeli wartości są mniejsze od 1 to kontrast jest zmniejszany. 

\subsection{Negatyw}
Wykonanie negatywu danego obrazu polega na przekształceniu wszystkich pixeli za pomocą wzoru:
\begin{equation}
 p(i) = i_{max} - i
\end{equation}
gdzie:
\begin{description}
\item $p(i)$ -- nowa wartość składowej RGB,
\item $i_{max}$ -- maksymalna dopuszczalna wartość,
\item i -- wartość konkretnej składowej danego piksela.
\end{description}

\subsection{Filtr ze średnią arytmetyczną}
Filtracja za pomocą tego algorytmu polega na wybraniu wielkości okna maski (w środku okna znajduje się piksel aktualnie przetwarzany). Kolejnym krokiem jest uśrednienie wartości poszczególnych pikseli w danym oknie. Dla obrazu kolorowego algorytm wykonujemy dla każdego kanału osobna. Np. dla maski rozmiaru 3x3 uśredniane są wartości 9 pikseli.

\subsection{Filtr medianowy}
Algorytm podobny do filtracji ze średnią arytmetyczną jednak zamiast uśredniana wartości poszczególnych pikseli w masce wybierana jest ich mediana czyli wartość środkowa z uszeregowanego rosnąco ciągu wszystkich wartości. Dla obrazu kolorowego algorytm również wykonywany jest dla każdego kanału osobno.

\subsection{Generowanie i modyfikacja histogramu}
Aby wygenerować histogram należy zebrać informacje o tym ile pikseli w danym kolorze znajduje się na obrazie (dla obrazu kolorowego każdy kanał badamy osobno). Po zsumowaniu wszystkich pikseli generowany jest wykres słupkowy gdzie na osi X znajdują się kolejne wartości jakie może przyjąć kolor (zwykle jest to 0-255) a na osi Y ilość pikseli w danym kolorze na obrazie.

Dla obrazu w odcieniach szarości generowany jest jeden histogram natomiast dla kolorowego cztery. Trzy z nich to kolejne kanały obrazu (red, green, blue) a jeden dodatkowy to histogram wartości luminacji dla danego obrazu liczona ze wzoru:
\begin{equation}
 y = 0,299r + 0,587g + 0,114b, \label{luminacja}
 \end{equation}
gdzie:
\begin{description}
\item y -- wartość luminacji dla danego piksela,
\item r -- wartość kanału red dla danego piksela,
\item g -- wartość kanału green dla danego piksela,
\item b -- wartość kanału blue dla danego piksela.
\end{description}

\subsection{Filtracja liniowa - identyfikowanie linii}

\subsection{Filtracja nieliniowa - operator Rosenfelda}

\subsection{Miary podobieństwa}
Błąd średniokwadratowy (MSE) określony jest wzorem:
\begin{equation}
 MSE = \frac{1}{N M}\sum\limits_{i=1}^N \sum\limits_{j=1}^M ([f(i,j)-f'(i,j)]^2) 
\end{equation}
gdzie:
\begin{description}
 \item $N,M$ -- wymiary obrazka
 \item $f(x,y)$ -- wartość piksela obrazu wzorcowego
 \item $f'(x,y)$ -- wartość piksela obrazu badanego
\end{description}

Szczytowy stosunek sygnału do szumu (PSNR) wyrażony w dB:
\begin{equation}
 PSNR = 10 \log_{10}\frac{k^2}{MSE}
\end{equation}
gdzie:
\begin{description}
 \item $k$ -- liczba kolorów obrazu minus 1(w naszym przypadku 255)
\end{description}


\section{Opis implementacji}


\section{Materiały i metody}
W tym miejscu należy opisać, jak przeprowadzone zostały wszystkie badania,
których wyniki i dyskusja zamieszczane są w dalszych sekcjach. Opis ten
powinien być na tyle dokładny, aby osoba czytająca go potrafiła wszystkie
przeprowadzone badania samodzielnie powtórzyć w celu zweryfikowania ich
poprawności (a zatem m.in. należy zamieścić tu opis architektury sieci,
wartości współczynników użytych w kolejnych eksperymentach, sposób
inicjalizacji wag, metodę uczenia itp. oraz informacje o danych, na których
prowadzone były badania). Przy opisie należy odwoływać się i stosować do
opisanych w sekcji drugiej wzorów i oznaczeń, a także w jasny sposób opisać
cel konkretnego testu. Najlepiej byłoby wyraźnie wyszczególnić (ponumerować)
poszczególne eksperymenty tak, aby łatwo było się do nich odwoływać dalej.

\section{Wyniki}
W tej sekcji należy zaprezentować, dla każdego przeprowadzonego eksperymentu,
kompletny zestaw wyników w postaci tabel, wykresów itp. Powinny być one tak
ponazywane, aby było wiadomo, do czego się odnoszą. Wszystkie tabele i wykresy
należy oczywiście opisać (opisać co jest na osiach, w kolumnach itd.) stosując
się do przyjętych wcześniej oznaczeń. Nie należy tu komentować i interpretować
wyników, gdyż miejsce na to jest w kolejnej sekcji. Tu również dobrze jest
wprowadzić oznaczenia (tabel, wykresów) aby móc się do nich odwoływać
poniżej.

\section{Dyskusja}
Sekcja ta powinna zawierać dokładną interpretację uzyskanych wyników
eksperymentów wraz ze szczegółowymi wnioskami z nich płynącymi. Najcenniejsze
są, rzecz jasna, wnioski o charakterze uniwersalnym, które mogą być istotne
przy innych, podobnych zadaniach. Należy również omówić i wyjaśnić wszystkie
napotakane problemy (jeśli takie były). Każdy wniosek powinien mieć poparcie
we wcześniej przeprowadzonych eksperymentach (odwołania do konkretnych
wyników). Jest to jedna z najważniejszych sekcji tego sprawozdania, gdyż
prezentuje poziom zrozumienia badanego problemu.

\section{Wnioski}
W tej, przedostatniej, sekcji należy zamieścić podsumowanie
najważniejszych wniosków z sekcji poprzedniej. Najlepiej jest je po prostu
wypunktować. Znów, tak jak poprzednio, najistotniejsze są wnioski o
charakterze uniwersalnym.

\begin{thebibliography}{8}

\end{thebibliography}

\end{document}
