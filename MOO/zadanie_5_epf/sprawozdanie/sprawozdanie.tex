% vim:encoding=utf8 ft=tex sts=2 sw=2 et:


\documentclass{classrep}
\usepackage[utf8]{inputenc}

\usepackage[pdftex]{color,graphicx}
\DeclareGraphicsExtensions{.pdf,.png,.jpg}

\usepackage{float}

\usepackage{color}

\usepackage{listings}

\usepackage{color}

\usepackage[hyphens]{url}

\usepackage{hyperref}

\usepackage[polish]{babel}

\usepackage{enumitem}

\usepackage{amsmath}
\usepackage{fancyvrb}


\usepackage{indentfirst}
\usepackage[center,small,bf]{caption}
\hypersetup{colorlinks=false,pdfborder={0 0 0}}

\renewcommand{\labelitemi}{$\bullet$}
\renewcommand{\labelitemii}{$\cdot$}
\renewcommand{\labelitemiii}{$\diamond$}
\renewcommand{\labelitemiv}{$\ast$}
\definecolor{dkgreen}{rgb}{0,0.6,0}
\definecolor{gray}{rgb}{0.5,0.5,0.5}
\definecolor{mauve}{rgb}{0.58,0,0.82}
\lstset{ %
  language=Matlab,         % the language of the code
  basicstyle=\footnotesize,           % the size of the fonts that are used for the code
  numbers=left,     % where to put the line-numbers
  numberstyle=\tiny\color{gray},  % the style that is used for the line-numbers
  stepnumber=2,     % the step between two line-numbers. If it's 1, each line 
             % will be numbered
  numbersep=5pt,           % how far the line-numbers are from the code
  backgroundcolor=\color{white},      % choose the background color. You must add \usepackage{color}
  showspaces=false,        % show spaces adding particular underscores
  showstringspaces=false,         % underline spaces within strings
  showtabs=false,          % show tabs within strings adding particular underscores
  frame=single,     % adds a~frame around the code
  rulecolor=\color{black},        % if not set, the frame-color may be changed on line-breaks within not-black text (e.g. commens (green here))
  tabsize=2,        % sets default tabsize to 2 spaces
  captionpos=b,     % sets the caption-position to bottom
  breaklines=true,         % sets automatic line breaking
  breakatwhitespace=false,        % sets if automatic breaks should only happen at whitespace
  title=\lstname,     % show the filename of files included with \lstinputlisting;
             % also try caption instead of title
  keywordstyle=\color{blue},          % keyword style
  commentstyle=\color{dkgreen},       % comment style
  stringstyle=\color{mauve},         % string literal style
  escapeinside={\%*}{*)},     % if you want to add a~comment within your co
}

\studycycle{Informatyka, studia dzienne, mgr II st.}
\coursesemester{I}

\coursename{Metody obliczeniowe optymalizacji}
\courseyear{2011/2012}

\courseteacher{mgr inż. Łukasz Chomątek}
\coursegroup{czwartek, 16:00}

\author{
  \studentinfo{Paweł Musiał}{178726} \and
  \studentinfo{Łukasz Michalski}{178724}
}

\title{Zadanie 5 - Optymalizacja wielowymiarowa z ograniczeniami}
\svnurl{https://serce.ics.p.lodz.pl/svn/labs/moo/lc_cz1600/lmpm}

\begin{document}
\maketitle

\section{Cel}

Napisać program rozwiązujący problem optymalizacji z ograniczeniami za pomocą jednej z poniższych metod:
\begin{itemize}
\item \textbf{zewnętrzna lub wewnętrzna funkcja kary,}
\item SLP,
\item \textbf{optymalizacja wielokryterialna - metoda sumy ważonej,}
\item Box.
\end{itemize}


\section{Rozwiązanie zadania}

\subsection{Optymalizacja wielokryterialna - metoda sumy ważonej}

Problemy wielokryterialne, to problemy w których optymalizowanym wyrażeniem, jest parę funkcji celu. Jednym z~proponowanych rozwiązań w~literaturze \cite{2} do tego typu problemów jest metoda sumy ważonej. Metoda ta polega na zastąpieniu wielu funkcji jedną funkcją będącą liniową kombinacją składowych.

\noindent Dla przykładu, mając problem postaci:
\begin{equation}
min F(f_1, f_2, f_3, \dots, f_n))
\end{equation}
\noindent przy ograniczeniach :
\begin{align}
g(x) \leq 0 \nonumber\\
h(x) = 0 \nonumber
\end{align}
zastępujemy grupę funkcji celu za pomocą :
\begin{equation}
U = \sum \limits ^{n} _{i=1} w_i \cdot f_i
\end{equation}
\noindent gdzie : 
\begin{equation*}
\sum \limits ^{n} _{i=1} w_i = 1
\end{equation*}
oraz $w>0$.

Rozwiązanie takiego problemu będzie leżało na tak zwanym froncie pareto. Jest to powierzchnia na której występują rozwiązania niedominujące, czyli rozwiązań w sensie pareto, co oznacz, że każde rozwiązanie w~zbiorze jest optymalne i~nie lepsze ani gorsze od pozostałych.
 
Metoda ta jedynie opisuje sposób w jaki można rozwiązać problem z wieloma funkcjami celu, nie precyzuje jednak sposobu dalszych rozwiązań   wyznaczonego problemu zastępczego $U$. Wartości wag w~tej metodzie, można dobierać według potrzeb, oczywiście pamiętając o~nakreślonych restrykcjach. Z~drugiej strony zostały zaproponowanie rozwiązania adaptacyjnego doboru wag.

Jako, że metoda ta jedynie mówi o podejściu jakie można zastosować do problemów wielokryterialnych, polegającym na złożeniu wielu funkcji bazowych w jedną funkcje będącą superpozycją składowych przemnożonych przez jakieś wartości wag, tak na prawdę każdy bardziej złożony problem możemy traktować jako takie właśnie rozwiązanie. Dlatego też w~naszym programie skupiliśmy się na jednej z metod rozwiązywania problemu ,,zastępczego'' czyli problemu wielowymiarowej optymalizacji z ograniczeniami.

\subsection{Zewnętrzna funkcja kary}
Jest to metoda rozwiązywania problemów wielokryterialnych z ograniczeniami. Metoda ta jest prosta i polega na rozwiązywaniu problemu zastępczego, w~którym rozwiązujemy problem podstawowy z~dodaną funkcją kary będącą wartością - karą, za niespełnienie ograniczeń. Niestety nie przedstawimy tutaj dowodu poprawności tej metody. Jednak w~wielu przypadkach metoda ta pozwala na dobre wyniki. Przedstawiony poniżej opis jest zaczerpnięty z~\cite{3}.

\noindent Minimalizacja $f(x)$, przy ograniczeniach:
\begin{align}
[g(x)]_m\leq0 \nonumber\\
 [h(x)]_l=0 \nonumber
\end{align}
\noindent Wprowadzamy problem zastępczy, postaci: 
\begin{equation}
F(x,r_h, r_g) = f(x) + P(x,r_h, r_g)
\end{equation}
\noindent  gdzie : 
\begin{equation}
P(x,r_h, r_g)=r_h \left[ \sum \limits ^{l} _{j=1} (h_j(x))^{2} \right]  + r_g \left[ \sum \limits ^{m} _{j=1}
 \left( max \left( 0, g_j (x) \right) \right) ^{2} \right]
\end{equation}

Gdzie parametry $r_h$ i $r_g$ mówią, jak ,,mocno'' karać za naruszenie ograniczeń. Podstawiając wszystko do jednego równania otrzymujemy :
\begin{equation}
 F(x,r_h, r_g) = f(x) + r_h \left[ \sum \limits ^{l} _{j=1} (h_j(x))^{2} \right]  + r_g \left[ \sum \limits ^{m} _{j=1}
 \left( max \left( 0, g_j (x) \right) \right) ^{2} \right]
\end{equation}
Jak widać niejako pozbyliśmy się ograniczeń włączając je do minimalizowanego wyrażenia. W~sposób ten otrzymaliśmy problem wielowymiarowej optymalizacji bez ograniczeń który możemy rozwiązywać znanymi nam metodami czyli na przykład algorytmem pełzającego sympleksu, bądź jedną z metod quasi-newtonowskich.

Algorytm w opisie krokowym:
\begin{enumerate}
\item Wybierz punkt początkowy $x=x_0$ znajdujący się w obszarze dopuszczalnym, współczynniki kary $r_h$, $r_g$ i odpowiadające im współczynniki skalujące $C_{r_h}$ i $C_{r_g}$, oraz kryterium stopu, tutaj ilość iteracji $k_{max}$, zainicjuj licznik iteracji $k=1$.
\item Rozwiąż problem zastępczy   $x_q = min \left( F(x,r_g, r_h) \right)$
\item Sprawdź czy ograniczenia zostały naruszone $g_i(x_q)\leq0$, $h_j(x_q)=0, i \in \left(1 \dots m \right) , j \in \left(1 \dots l \right)$, jeśli ograniczenia nie zostały naruszone, sprawdź kryterium stopu.
\item Skaluj współczynniki kary : $r_h=r_h \cdot C_{r_h}$, $r_g=r_g \cdot C_{r_g}$
\item Zwiększ licznik iteracji $k=k+1$, przypisz $x=x_q$ przejdź do 2.
\end{enumerate}

\section{Opis programu}

\begin{lstlisting}{Ograniczenia równościowe}
function fstr=equalConSum(f)
fstr='( ';
for i=1:length(f)-1
    fstr=strcat(fstr,'',f(i),')^2 + ');
end
fstr=strcat(fstr,'(',f(length(f)),')^2 )');
\end{lstlisting}


\begin{lstlisting}{Ograniczenia nierównościowe}
function fstr=inEqualConSum(f)
fstr='( ';
for i=1:length(f)-1
    fstr=strcat(fstr,'(max(0,',f(i),'))^2 + ');
end
fstr=strcat(fstr,'(max(0,',f(length(f)),'))^2 )');
\end{lstlisting}


\begin{lstlisting}{Funkcja problemu zastępczego}
function fout = penaltyFun( f, h, g, rh, rg )
hfun=0;
gfun=0;
if ~isempty(h)
    hfun=@(x,y)subs(h,{'x','y'},{x,y});
end
if ~isempty(g)
    gfun=@(x,y)subs(g,{'x','y'},{x,y});
end

ffun=@(x,y)subs(f,{'x','y'},{x,y});

fout = @(x) ffun(x(1),x(2)) + rh*hfun(x(1),x(2)) +
 rg*gfun(x(1),x(2));

end

\end{lstlisting}

\begin{lstlisting}{Algorytm zewnętrznej funkcji kary}
function [xi, fval] = epf(f, h, g, rh, rg, x0, Crh, Crg, kMax )
hpenalty='';
gpenalty='';
if ~isempty(h)
    hpenalty=equalConSum(h);
end
if ~isempty(g)
    gpenalty=inEqualConSum(g);
end
xi=x0
for i=1:kMax

    Fp=penaltyFun(f,hpenalty,gpenalty,rh,rg);
    
    [xi, fval]=fminunc(Fp,xi)
    unsatisfied=0;
    for j=1:length(h)
        hc=subs(h(j),{'x','y'},{xi(1),xi(2)});
        if (hc~=0)
            unsatisfied=unsatisfied+1;
        end
    end
    for j=1:length(g)
        gc=subs(char(g(j)),{'x','y'},{xi(1),xi(2)});
        if (gc>0)
            unsatisfied=unsatisfied+1;
        end
    end
    if(unsatisfied==0 || i>=kMax)
        break;
    end
    rh=rh*Crh;
    rg=rg*Crg;
end
end
\end{lstlisting}

\section{Wyniki}

Rozwiązanie problemu testowano na funkcji przykładowej zawartej w~książce \cite{3}.
\begin{equation}
f(x_1, x_2) = x^4 _1 - 2 x^2 _1 x_2 + x^2 _1 + x_1 x^2 _2 - 2x_1 +4
\end{equation}

\noindent przy ograniczeniach:
\begin{equation*}
h(x_1, x_2): x^2 _1 + x^2 _2 -2 = 0
\end{equation*}
\begin{equation*}
g(x_1, x_2: 0.25x^2 _1 + 0.75 x^2 _2 -1 \leq 0
\end{equation*}

\noindent przy dobranych parametrach:
\begin{enumerate}
\item $x_0 = [1, 2]$
\item $k_{max}=3$
\item $r_h=5,\ C_{r_h}=0.95$
\item $r_g=5,\ C_{r_g}=0.95$
\end{enumerate}

\begin{center}
\begin{tabular}{|c|c|c|}
\hline 
iter & x & f(x)\\
\hline
1 & 0.9464, 1.0364 & 2.9740\\ 
\hline 
2 &  0.9454, 1.0369 & 2.9736\\ 
\hline 
3 & 0.9445, 1.0373 & 2.9731\\ 
\hline 
\end{tabular} 
\end{center}


Wynik w \cite{2} dla porównania wynosił x=[0.9775, 1.0165], f(x)=2.9810.
Jak widać uzyskany wynik jest bardzo bliski wartości otrzymanej przez obliczonej przez naszą implementacje, tejże metody już po 3 iteracjach.

\section{Wnioski}
Przedstawiona powyżej implementacja optymalizacji wielowymiarowej z~ograniczeniami przy pomocy zewnętrznej funkcji kary jest jedną z~metod optymalizacji znajdujących rozwiązanie rozwiązując przybliżone problemy do oryginalnego. Podejście takie skutkuje dobrymi wynikami i~jest łatwe w~implementacji ponieważ opierają się one z~reguły na algorytmach rozwiązywania standardowych problemów optymalizacji, mamy tutaj na uwadze takie algorytmy jak SLP, SQP, metody oparte o funkcję kary. Jedynym minusem ich działania jest nie działanie na prawdziwej funkcji celu a~jedynie na jej przybliżeniu, co w~konsekwencji niejako implikuje słabą jakość przy zadaniach bardziej złożonych. Zaimplementowana przez nad metoda testowana była na prostym przykładzie dlatego też szybko uzyskała zbieżność.

\begin{thebibliography}{9}
\bibitem{1} ,,An Introduction to Optimization'' , Edwin Kah Pin Chong and Stanislaw H. Zak, Hoboken, EUA : Wiley-Interscience 2008
\bibitem{2} ,,Introduction to Optimum Design'', Arora, Jasbir S., Elsevier Academic Press 2004
\bibitem{3} ,,Applied Optimization with MATLAB Programming'', Venkatamaran P., Wiley 2001
\end{thebibliography}
\end{document}