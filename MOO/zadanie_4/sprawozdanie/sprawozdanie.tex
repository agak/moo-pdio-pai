% vim:encoding=utf8 ft=tex sts=2 sw=2 et:


\documentclass{classrep}
\usepackage[utf8]{inputenc}

\usepackage[pdftex]{color,graphicx}
\DeclareGraphicsExtensions{.pdf,.png,.jpg}

\usepackage{float}

\usepackage{color}

\usepackage{listings}

\usepackage{color}

\usepackage[hyphens]{url}

\usepackage{hyperref}

\usepackage[polish]{babel}

\usepackage{enumitem}

\usepackage{amsmath}
\usepackage{fancyvrb}


\usepackage{indentfirst}
\usepackage[center,small,bf]{caption}
\hypersetup{colorlinks=false,pdfborder={0 0 0}}

\renewcommand{\labelitemi}{$\bullet$}
\renewcommand{\labelitemii}{$\cdot$}
\renewcommand{\labelitemiii}{$\diamond$}
\renewcommand{\labelitemiv}{$\ast$}
\definecolor{dkgreen}{rgb}{0,0.6,0}
\definecolor{gray}{rgb}{0.5,0.5,0.5}
\definecolor{mauve}{rgb}{0.58,0,0.82}
\lstset{ %
  language=Matlab,         % the language of the code
  basicstyle=\footnotesize,           % the size of the fonts that are used for the code
  numbers=left,     % where to put the line-numbers
  numberstyle=\tiny\color{gray},  % the style that is used for the line-numbers
  stepnumber=2,     % the step between two line-numbers. If it's 1, each line 
             % will be numbered
  numbersep=5pt,           % how far the line-numbers are from the code
  backgroundcolor=\color{white},      % choose the background color. You must add \usepackage{color}
  showspaces=false,        % show spaces adding particular underscores
  showstringspaces=false,         % underline spaces within strings
  showtabs=false,          % show tabs within strings adding particular underscores
  frame=single,     % adds a~frame around the code
  rulecolor=\color{black},        % if not set, the frame-color may be changed on line-breaks within not-black text (e.g. commens (green here))
  tabsize=2,        % sets default tabsize to 2 spaces
  captionpos=b,     % sets the caption-position to bottom
  breaklines=true,         % sets automatic line breaking
  breakatwhitespace=false,        % sets if automatic breaks should only happen at whitespace
  title=\lstname,     % show the filename of files included with \lstinputlisting;
             % also try caption instead of title
  keywordstyle=\color{blue},          % keyword style
  commentstyle=\color{dkgreen},       % comment style
  stringstyle=\color{mauve},         % string literal style
  escapeinside={\%*}{*)},     % if you want to add a~comment within your co
}

\studycycle{Informatyka, studia dzienne, mgr II st.}
\coursesemester{I}

\coursename{Metody obliczeniowe optymalizacji}
\courseyear{2011/2012}

\courseteacher{mgr inż. Łukasz Chomątek}
\coursegroup{czwartek, 16:00}

\author{
  \studentinfo{Paweł Musiał}{178726} \and
  \studentinfo{Łukasz Michalski}{178724}
}

\title{Zadanie 4 - Programowanie liniowe} % np \title{Zadanie 1: Optymalizacja jednowymiarowa}
\svnurl{https://serce.ics.p.lodz.pl/svn/labs/moo/lc_cz1600/lmpm}

\begin{document}
\maketitle

\section{Cel}
Napisać program implementujący rozwiązywanie zagadnienia programowania liniowego za pomocą dwufazowej metody sympleksu. Program powinien wykrywać sytuacje patologiczne (brak rozwiązań, nieskończenie wiele rozwiązań).

\section{Rozwiązanie zadania}

\subsection{Problem programowania liniowego}

Programowaniem liniowym nazywamy problem optymalizacji wielowymiarowej funkcji liniowej z ograniczeniami.
Zapisać możemy go jako minimalizacja wyrażenia :
\begin{equation}
c^{T}x
\end{equation}
z ograniczeniami :

\begin{eqnarray}
Ax=b \\
x \geq 0
\end{eqnarray}
Zakładając, że $b \geq 0$\\
gdzie :
\begin{itemize}
\item x - zmienne funkcji celu
\item c - współczynniki funkcji celu
\item A - macierze współczynników ograniczeń
\item b - prawa strona ograniczeń
\end{itemize}

Przedstawioną powyżej postać, nazywamy postacią standardową, aby z dowolnej postaci problemu programowania liniowego
\begin{eqnarray}
c^{T}x\\
Ax \geq b \\
x \geq 0
\end{eqnarray}

 przejść do tej postaci należy : w przypadku maksymalizacji, przekształcamy zadanie na minimalizacje funkcji przeciwnej funkcji celu; w związku z założoną nie ujemnością prawej strony ograniczeń, jeśli wartość $b_i$ jest ujemna, przemnażamy ograniczenie przez $-1$. Następnie wprowadzamy sztuczne zmienne $y_i$. A problem w standardowej postawi przyjmuje się jako :
\begin{eqnarray}
c^{T}x\\
\left[ \begin{smallmatrix} A &,w -I_m \end{smallmatrix} \right] \left[ \begin{smallmatrix} x\\y \end{smallmatrix} \right] = b\\
y \geq 0
\end{eqnarray}
gdzie $I_m$ jest macierzą jednostkową o wymiarach $mxm$\\
lub gdy ograniczenia przybierają formę : $Ax \geq b$\\
ograniczenia w standardowej formie przybierają formę : 
$\left[ \begin{smallmatrix} A &, I_m \end{smallmatrix} \right] \left[ \begin{smallmatrix} x\\y \end{smallmatrix} \right] = b$\\


\subsection{Metoda sympleks}

Znając już postać problemu możemy przejść do opisu rozwiązania. 
W algorytmie tym układ równań liniowych $Ax=b$ (gdzie $rz(A)=m$) zapisujemy w postaci ,,kanonicznej'', którą możemy otrzymać na drodze przekształceń elementarnych :\\
$$\left[ \begin{smallmatrix} I_m &, Y_{m,n-m} \end{smallmatrix} \right] x = y_0 $$
Piersze m zmiennych $x$ występuje tylko w jednym równaniu, a współczynniki przy nich są równe 1, zmienne ta nazywamy zmiennymi bazowymi, w innym przypadku niebazowymi.


Związana z tą postacią jest macierz kanoniczna : $\left[  I_m , Y_{m,n-m}, y_0 \right] $.

\begin{enumerate}
\item Rozpoczynamy od pewnego dopuszczalnego rozwiązania x= $\left[ x_1, x_2,...,x_m,0,...,0 \right]^T$ gdzie $x_i \geq 0 $ dla $i \in 1..m$. Wyznaczamy macierz kanoniczną dla tego rozwiązania.
\item Obliczamy współczynniki zredukowanego kosztu względem każdej zmiennej nie bazowej -  $r_i = c_i - z_i $ gdzie $z_i = c_1 y_{1i}+ \dots + c_m y_{mi}$.
\item Jeśli $\forall\ j\ r_j \geq 0$ to aktualne rozwiązanie jest optymalne. Koniec.
\item Wybierze q dla którego $r_q  < 0$.
\item Jeśli  nie istnieje $y_{iq} >0$ problem jest nieograniczony - koniec. W innym przypadku oblicz $p = min \{ j : \frac{y_{j0}}{y_{jq}} = min_i \{ \frac{y_{i0}}{y_{iq}} : y_{iq} > 0 \} \}$ .
\item zmień bazę zmiennych p i q w następujący sposób:
\begin{eqnarray}
y'_{ij} = y_{ij} - \frac{y_{pj}}{y_{pq}} y_{iq},\ dla\ i \neq p\\
y'_{pj} = \frac{y_{pj}}{y_{pq}}
\end{eqnarray}
\item Wróć do kroku 2.
\end{enumerate}
Powyższy algorytm opisuje kolejne kroki metody rozwiązywania problemów programowania liniowego.

Jak już powyżej opisano, zmienne w ograniczeniach rozdzielamy na bazowe i niebazowe. Na podstawie tego rozgraniczenia możemy zapisać problem programowania liniowego, oznaczając cześć dotyczącą zmiennych bazowych poprzez $B$ a niebazowych przez $D$.\\
Możemy teraz opisać zadanie jako :
\begin{eqnarray}
min\ c^{T} _{B} x_{B} + c^{T}_{D}x_{D}\\
\left[ \begin{smallmatrix} B &, D \end{smallmatrix} \right] \left[ \begin{smallmatrix} x_B\\x_D \end{smallmatrix} \right] =b \\
x_{B} \geq 0, x_{D} \geq 0
\end{eqnarray}

Jeśli wektor $x_D \neq 0$, czyli aktualne rozwiązani nie jest optymalnym, możemy wyznaczyć wektor kosztów zredukowanych :
\begin{equation}
r^{T}_{D} = c^{T}_{D} - c^{T}_{B} B^{-1} D
\end{equation}

Możemy teraz przedstawić postać macierzy sympleksowej :
\begin{equation}
\left[ \begin{smallmatrix} A & b \\ c^{T} & 0  \end{smallmatrix} \right] =
\left[ \begin{smallmatrix} B &  D & b \\c^{T}_{B} & c^{T}_{D} \end{smallmatrix} \right]  \\
\end{equation}

Algorytm sympleksu działa na macierzy kanonicznej, należy sprowadzić zatem macierz sympleksową do tej formy, po przekształceniach, które można dokładnie poznać w  \cite{1}. A w wyniku otrzymujemy :
\begin{equation}
	\begin{bmatrix}
		I_m & B^{-1} D & B^{-1} b \\
		0^{T} & c^{T}_{D} - c^{T}_{B} B^{-1} D & -c^{T} _{B} B^{-1} b   \\
	\end{bmatrix}
\end{equation}

Rozważmy teraz przykład :\\
maksymalizacja wyrażenia $7x_1 + 6x_2$\\
z ograniczeniami $2x_1 + x_2 \leq 3$, $x_1 + 4x_2 \leq 4$, $x_1,x_2\geq 0$

Zadaniem jest maksymalizacja, zatem w tablicy sympleksowej wspólczynniki funkcji celu $c$ będą zanegowane, oraz wprowadzone zostaną dwie nieujemne sztuczne zmienne. Zatem tablica sympleksowa będzie miała postać :\\
\begin{tabular}{|c|c|c|c|c|}
\hline 
2 & 1 & 1 & 0 & 3 \\ 
\hline 
1 & 4 & 0 & 1 & 4 \\ 
\hline 
-7 & -6 & 0 & 0 & 0 \\ 
\hline 
\end{tabular} 
\\
Zauważmy, że tablica ta jest już w postaci kanonicznej, wyznaczamy min $r_i = r_1 = -7 \implies$ q=1, następnie wyznaczamy p zgodnie z algorytmem - p=1. Zatem wykonujemy operacje zmiany bazy (1,1). \\
\begin{tabular}{|c|c|c|c|c|}
\hline 
1 & $\frac{1}{2}$ & $\frac{1}{2}$ & 0 & $\frac{3}{2}$ \\ 
\hline 
0 & $\frac{7}{2}$ & $-\frac{1}{2}$ & 1 & $\frac{5}{2}$ \\ 
\hline 
0 & $-\frac{5}{2}$ & $\frac{7}{2}$ & 0 & $\frac{21}{2}$ \\ 
\hline 
\end{tabular} 
\\
Jedynie $r_2$ jest ujemne, wybieramy q=2, wybieramy zgodnie z algorytmem p=2, Wykonujemy operacje zmiany bazy (2,2).\\

\begin{tabular}{|c|c|c|c|c|}
\hline 
1 & 0 & $\frac{4}{7}$ & $-\frac{1}{7}$ & $\frac{8}{7}$ \\ 
\hline 
0 & 1 & $-\frac{1}{7}$ & $\frac{2}{7}$ & $\frac{5}{7}$ \\ 
\hline 
0 & 0 & $\frac{22}{7}$ & $\frac{5}{7}$ & $\frac{86}{7}$\\ 
\hline 
\end{tabular} 
\\
Ponieważ żadna wartość z ostatniego wiersza nie jest ujemna, kończymy działania algorytmu. wyznaczamy rozwiązanie jako $x_1 = \frac{8}{7}$ , $x_2=\frac{5}{7}$, oraz wartość funkcji celu $\frac{86}{7}$.

\subsection{Metoda dwu-fazowego sympleksu}

W poprzedniej sekcji, założyliśmy, że początkowe rozwiązanie jest dopuszczalne, jednak takie założenie może być błędne. Metoda dwu-fazowego sympleksu wyznacza w pierwszej fazie początkowe rozwiązanie dopuszczalne, a następnie oblicza na jego podstawie rozwiązanie podstawowego problemu.\\
Początkowe rozwiązanie dopuszczalne wyznaczamy rozwiązując zadanie pomocnicze.\\
Rozwiązujemy zadanie minimalizacji\\
 $y_1+y_2+\dots+y_m$\\
 przy ograniczeniach
 $\left[ \begin{smallmatrix} A &, I_m \end{smallmatrix} \right] \left[ \begin{smallmatrix} x\\y \end{smallmatrix} \right] = b$\\
$x, y \geq 0 $\\
gdzie $y=[y_1+y_2+\dots+y_m]^{T}$ jest wektorem zmiennych sztucznych, oraz wiemy, że początkowe rozwiązanie dopuszczalne ma postać $\left[ \begin{smallmatrix} 0\\b \end{smallmatrix} \right]$.
Po rozwiązaniu zadania pomocniczego możemy wyznaczyć tablice sympleksową w postaci kanonicznej problemu  podstawowego odrzucając sztuczne zmienne, oraz wracając do podstawowej funkcji celu.



\section{Opis programu}


\subsection{Algorytm podstawowy : metoda sympleks}

\begin{lstlisting}{simplex}
function [subs, A, z]= simplex(A, subs, mm, k)
[m, n] = size(A);
[mi, col] = BlandRule(A(m,1:n-1));
while ~isempty(mi) & mi < 0 & abs(mi) > eps
    t = A(1:m-k,col);
    if all(t <= 0)
        if mm == 0
     z = -inf;
        else
     z = inf;
        end
        fprintf('\nnieograniczone rozwiazanie z= %s\n',z)
        return
    end
    c = 1:m;
    a=A(1:m-k,n);
    b=A(1:m-k,col);
    l = c(b > 0);
    [small, row] = min(a(l)./b(l));
    row = l(row);
    if ~isempty(row)
        if abs(small) <= 100*eps & k == 1
     [s,col] = BlandRule(A(m,1:n-1));
        end
        A(row,:)= A(row,:)/A(row,col);
        subs(row) = col;
        for i = 1:m
     if i ~= row
         A(i,:)= A(i,:)-A(i,col)*A(row,:);
     end
        end
        [mi, col] = BlandRule(A(m,1:n-1));
    end
end
z = A(m,n);
end
\end{lstlisting}


\subsection{Dwu fazowy sympleks}

\begin{lstlisting}{two-phase simplex}
function  tpsimplex(type, c, A, rel, b)

if (type == 'min')
    mm = 0;
else
    mm = 1;
    c = -c;
end
c=c(:)';b=b(:);
[m, n] = size(A);
n1 = n;
les = 0;
if length(c) < n
    c = [c zeros(1,n-length(c))];
end
for i=1:m
    artificial_var =zeros(m,1);
    artificial_var(i)=1;
    if(rel(i) == '<')
        A = [A artificial_var];
        les = les + 1;
    elseif(rel(i) == '>')
        A = [A -artificial_var];
    end
end
ncol = length(A);

if les == m
    c = [c zeros(1,ncol-length(c))];
    A = [A;c];
    A = [A [b;0]];
    [subs, A, z] = simplex(A, n1+1:ncol, mm, 1);
else
    A = [A eye(m) b];
    if m > 1
        w = -sum(A(1:m,1:ncol));
    else
        w = -A(1,1:ncol);
    end
    c = [c zeros(1,length(A)-length(c))];
    A = [A;c];
    A = [A;[w zeros(1,m) -sum(b)]];
    subs = ncol+1:ncol+m;
    av = subs;
    [subs, A, z] = simplex(A, subs, mm, 2);
    
    nc = ncol + m + 1;
    x = zeros(nc-1,1);
    x(subs) = A(1:m,nc);
    xa = x(av);
    com = intersect(subs,av);
    if (any(xa) ~= 0)
        fprintf('Brak rozwiazan\n')
        return
    end
    A = A(1:m+1,1:nc);
    A =[A(1:m+1,1:ncol) A(1:m+1,nc)];
    [subs, A, z] = simplex(A, subs, mm, 1);
end

if (z == inf | z == -inf)
    return
end
[m, n] = size(A);
x = zeros(n,1);
x(subs) = A(1:m-1,n);
x = x(1:n1);
if mm == 0
    z = -A(m,n);
else
    z = A(m,n);
end

disp(x(1:n1))
disp(z)

t = find(A(m,1:n-1) == 0);
if length(t) > m-1
    fprintf('Problem ma nieskonczenie wiele rozwiazan\n');
end

end
\end{lstlisting}

\section{Wyniki}
\fvset{fontsize=\small}
Dla danych wejściowych :
\begin{Verbatim}[frame=lines
]
type = 'min';
c = [7 6];
A = [2 1;1 4];
rel = '<<';
b = [3 4];
\end{Verbatim}

\begin{Verbatim}[frame=lines
]
Tableau
       2       1       1       0       3       
       1       4       0       1       4       
      -7      -6       0       0       0 
wymiana bazy 1  -> 1
Tableau
       1       1/2     1/2     0       3/2     
       0       7/2           -1/2     1       5/2     
       0      -5/2     7/2     0      21/2     
wymiana bazy 2  -> 2
Tableau
       1       0       4/7           -1/7     8/7     
       0       1      -1/7     2/7     5/7     
       0       0      22/7     5/7           86/7     
Rozwiazanie optymalne:
       8/7     
       5/7     
wartosc funkcji celu:
      86/7    
\end{Verbatim}


Dla danych wejściowych :


\begin{Verbatim}[frame=lines
]

type = 'min';
c = [7 6];
A = [2 1;1 4];
rel = '<<';
b = [3 4];
\end{Verbatim}

\begin{Verbatim}[frame=lines
]

Tableau
       2      -1      1      6     -5     -1      0      1      0      6 
       1       1      2      1      2      0     -1      0      1      3
       3       4      6      7      1      0      0      0      0      0
      -3       0    - 3     -7      3      1      1      0      0     -9 
wymiana bazy 1  -> 1
Tableau
       1      -1/2     1/2     3      -5/2    -1/2     0       1/2     0       3
       0       3/2     3/2    -2       9/2     1/2    -1      -1/2     1       0
       0      11/2     9/2    -2      17/2     3/2     0      -3/2     0      -9
       0      -3/2    -3/2     2      -9/2    -1/2     1       3/2     0       0
wymiana bazy 2  -> 2
Tableau
       1       0       1       7/3    -1      -1/3   -1/3     1/3     1/3     3
       0       1       1      -4/3     3       1/3   -2/3    -1/3     2/3     0
       0       0      -1      16/3    -8      -1/3   11/3     1/3   -11/3     -9
       0       0       0       0       0       0       0       1       1       0
Koniec fazy 1
Tableau
       1       0       1       7/3    -1      -1/3    -1/3     3       
       0       1       1      -4/3     3       1/3    -2/3     0       
       0       0      -1      16/3    -8      -1/3    11/3    -9       
wymiana bazy 2  -> 3
Tableau
       1      -1       0      11/3    -4     -2/3     1/3     3       
       0       1       1      -4/3     3      1/3    -2/3     0       
       0       1       0       4      -5      *         3	  -9       
wymiana bazy 2  -> 5
Tableau
       1       1/3     4/3     17/9     0     -2/9     -5/9      3       
       0       1/3     1/3     -4/9     1      1/9     -2/9      0       
       0       8/3     5/3     16/9     0      5/9     17/9     -9       
Koniec fazy 2

Rozwiazanie optymalne:
       3       
       0       
       0       
       0       
       0       
       0       
       0       

 wartosc funkcji celu:
       9    
\end{Verbatim}

\section{Wnioski}
Przedstawiona powyżej implementacja algorytmu dwu-fazowego sympleksu spełnia swoje zadanie jako metody rozwiązywania problemów programowania liniowego. Rozstrzyga również sytuacje trudne, takie jak : nieskończenie wiele rozwiązań, brak rozwiązań. Również zastosowana modyfikacja podstawowego sposobu doboru zmiennych do operacji zmiany bazy - ,,regułą Blanda'' pozwala uniknąć zablokowania się algorytmu podstawowego - metody sympleksu.

\begin{thebibliography}{9}
\bibitem{1} ,,An Introduction to Optimization'' , Edwin Kah Pin Chong and Stanislaw H. Zak, Hoboken, EUA : Wiley-Interscience 2008
\end{thebibliography}
\end{document}