\documentclass{classrep}
\usepackage[utf8]{inputenc}

\usepackage[pdftex]{color,graphicx}
\DeclareGraphicsExtensions{.pdf,.png,.jpg,.bmp}

\usepackage{float}
\usepackage{subfig}
\usepackage{color}
\usepackage{tabularx}

\usepackage{listings}

\usepackage{color}

\usepackage{url}

\usepackage{hyperref}

\usepackage[polish]{babel}

\usepackage{enumitem}
\usepackage{paralist}
\usepackage{indentfirst}

\hypersetup{colorlinks=false,pdfborder={0 0 0}}


\definecolor{javared}{rgb}{0.6,0,0} % for strings
\definecolor{javagreen}{rgb}{0.25,0.5,0.35} % comments
\definecolor{javapurple}{rgb}{0.5,0,0.35} % keywords
\definecolor{javadocblue}{rgb}{0.25,0.35,0.75} % javadoc

\lstset{language=Java,
basicstyle=\ttfamily,
keywordstyle=\color{javapurple}\bfseries,
stringstyle=\color{javared},
commentstyle=\color{javagreen},
morecomment=[s][\color{javadocblue}]{/**}{*/},
numbers=left,
numberstyle=\tiny\color{black},
stepnumber=2,
numbersep=10pt,
tabsize=4,
showspaces=false,
showstringspaces=false}


\studycycle{Informatyka, studia dzienne, mgr II st.}
\coursesemester{I}

\coursename{Przetwarzanie obrazu i dźwięku}
\courseyear{2011/2012}

\courseteacher{dr inż. Arkadiusz Tomczyk}
\coursegroup{środa, 8:15}

\author{
  \studentinfo{Paweł Musiał}{178726} \and
  \studentinfo{Łukasz Michalski}{178724}
}
\title{Zadanie 2:\\  \textbf {Filtracja w dziedzinie częstotliwości i segmentacja obrazu.}}
\svnurl{http://serce.ics.p.lodz.pl/svn/labs/poid/at_sr0800/lmpm}

\begin{document}
\maketitle

\addtocounter{footnote}{1}

\tableofcontents

\section{Cel}

Celem zadania było stworzenie aplikacji realizującej poniższe element :
\begin{itemize}
\item Zaimplementować proste i odwrotne szybkie przekształcenie Fouriera z decymacją w dziedzinie częstotliwości), a następnie zastosować je do obrazów. Powinna istnieć możliwość obejrzenia zarówno widma mocy, jak i widma fazy.
\item Zaimplementować następujące metody filtracji:
	\begin{itemize}
	\item	(F1) Filtr dolnoprzepustowy (górnozaporowy).
	\item	(F2) Filtr górnoprzepustowy (dolnozaporowy).
	\item	(F3) Filtr pasmowoprzepustowy.
	\item	(F4) Filtr pasmozaporowy.
	\item	(F5) Filtr z detekcją krawędzi.
	\end{itemize}
\item Zaimplementować filtr modyfikujący fazę widma transformaty Fouriera. Modyfikacja ta polega, dla obrazu o wymiarach N ×M , na pomnożeniu każdego elementu widma przez:
\begin{equation}
P(n,m) = \exp { \left( j \cdot \left( \frac{-n k 2 \pi}{N} + \frac{-m l 2 \pi }{M} \left( k + l  \right) \pi \right) }
\end{equation}
\item Zaimplementować metodę segmentacji - metoda podziału obszarów \footnote{ang. region splitting and merging}  w celu znalezienia spójnych obszarów o jednolitej barwie. Jako wynik należy wygenerować obrazy reprezentujące znaleziony obszary o jednolitej barwie (tyle masek ile znalezionych obszarów), na których kolorem białym oznaczone są te obszary, a czarnym pozostała część obrazu. Powinna istnieć możliwość nałożenia wybranych masek na obraz przy czym sposób wizualizacji tego nałożenia wybrany powinien zostać przez twórców aplikacji.


\end{itemize}


W sprawozdaniu zamieszczono wyniki działania poszczególnych algorytmów oraz porównanie ich pracy w różnych wariantach ustawień i~dla różnych problemów. Badania przeprowadzono zarówno na obrazach kolorowych (24-bitowych), jak i~w odcieniach szarości (8-bitowych), a~część badań również na obrazach czarno-białych (1-bitowych). 

\section{Wprowadzenie}
Obrazy przechowywane są w pamięci komputera w postaci bitowej, w której określa się ilość bitów przypadającą na każdy piksel obrazu. Kolorowe zdjęcia zapisywane są w formacie 24-bitowym co odpowiada 8-bitom na każdy~z trzech składowych kanałów formatu RGB. Fotografie w odcieniach szarości wykorzystują tylko jeden kanał dlatego wystarczy tutaj przechowywać 8-bitów na każdy zapisany piksel. Taki zapis umożliwia uzyskanie 256 różnych wartość danego piksela czyli 256 różnych odcieni szarości poczynając od białego na kolorze czarnym kończąc. W przypadku obrazów kolorowych liczba kombinacji jest dużo większa i wynosi 16,777,216, co nie oznacza, że wszystkie uzyskane w ten sposób kolory są między sobą rozróżnialne. W tej pracy wszystkie przygotowane algorytmy operują na poszczególnych wartościach każdego kanału obrazu.

\subsection{Filtracja w dziedzinie częstotliwości}

\subsubsection{Algorytm szybkiej transformaty Fouriera}

\subsubsection{Filtr dolnoprzepustowy}

\subsubsection{Filtr górnoprzepustowy}

\subsubsection{Filtr pasmowoprzepustowy}

\subsubsection{Filtr pasmowozaporowy}

\subsubsection{Filtr z detekcją krawędzi}

\subsubsection{Filtr modyfikujący widmo}

\subsection{Segmentacja}

\section{Opis implementacji}


\section{Materiały i metody}



\section{Wyniki}


\subsection{Filtracja w dziedzinie częstotliwości}

\subsubsection{Filtr dolnoprzepustowy}

\subsubsection{Filtr górnoprzepustowy}

\subsubsection{Filtr pasmowoprzepustowy}

\subsubsection{Filtr pasmowozaporowy}

\subsubsection{Filtr z detekcją krawędzi}

\subsubsection{Filtr modyfikujący widmo}


\subsection{Segmentacja}



\section{Dyskusja}



\section{Wnioski}


\begin{thebibliography}{8}

\end{thebibliography}

\end{document}
